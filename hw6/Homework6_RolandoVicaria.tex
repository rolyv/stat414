\documentclass{article}
    
    
    \usepackage{graphicx} % Used to insert images
    \usepackage{adjustbox} % Used to constrain images to a maximum size 
    \usepackage{color} % Allow colors to be defined
    \usepackage{enumerate} % Needed for markdown enumerations to work
    \usepackage{geometry} % Used to adjust the document margins
    \usepackage{amsmath} % Equations
    \usepackage{amssymb} % Equations
    \usepackage{eurosym} % defines \euro
    \usepackage[mathletters]{ucs} % Extended unicode (utf-8) support
    \usepackage[utf8x]{inputenc} % Allow utf-8 characters in the tex document
    \usepackage{fancyvrb} % verbatim replacement that allows latex
    \usepackage{grffile} % extends the file name processing of package graphics 
                         % to support a larger range 
    % The hyperref package gives us a pdf with properly built
    % internal navigation ('pdf bookmarks' for the table of contents,
    % internal cross-reference links, web links for URLs, etc.)
    \usepackage{hyperref}
    \usepackage{longtable} % longtable support required by pandoc >1.10
    \usepackage{booktabs}  % table support for pandoc > 1.12.2
    \usepackage{indentfirst}
    \usepackage{floatrow}
    \usepackage{relsize}
    
    \newcommand\perm[2]{{}_{#1}P_{#2}}%
    
    
    \title{Homework 6}
    \author{Roly Vicar\'ia \\ STAT414 Spring 2016}
    
\begin{document}
    
    \maketitle
    
    \textbf{Section 2.5}
    \begin{enumerate}
      %1
      \item 
	\begin{enumerate}
	 %a
	 \item 
	  $P(X \ge 13) = P(X > 12) = q^{12} = (0.9)^{12} = 0.2824$
	  
	 %b
	 \item
	  $P(X = 30) = \dbinom{29}{2}(0.1)^3(0.9)^{27} = 0.02361$
	\end{enumerate}

      
      %2
      \item
	$P(X = 10) = \dbinom{9}{4}(0.5)^5(0.5)^5 = 126\left(\dfrac{1}{2^10}\right) = \dfrac{63}{512}$
      \addtocounter{enumi}{2}
      
      %5
      \item
	\begin{enumerate}
	 %a
	 \item 
	  $R'(0) = \dfrac{M'(0)}{M(0)} = \dfrac{E(X)}{1} = \mu$
	  
	 %b
	 \item
	  $R''(0) = \dfrac{M(0)M''(0) - [M'(0)]^2}{[M(0)]^2} 
	    = \dfrac{(1)(\sigma^2 + [M'(0)]^2) - [M'(0)]^2}{1^2} 
	    = \dfrac{\sigma^2}{1} = \sigma^2$
	\end{enumerate}

      
      %6
      \item
	\begin{enumerate}
	 \addtocounter{enumii}{3}
	 %d
	 \item 
	  Using the result from 2.5-5, the mean, $\mu$, of the negative binomial distribution can be 
	  calculated as follows:
	  
	  \begin{align*}
	   \mu = R'(0) = \dfrac{M'(0)}{M(0)} = \dfrac{rp^{-1}}{1} = \dfrac{r}{p}
	  \end{align*}
	  
	  And variance, $\sigma^2$, can be calculated as follows:
	  
	  \begin{align*}
	   \sigma^2 = R''(0) &= \dfrac{M(0)M''(0) - [M'(0)]^2}{[M(0)]^2} \\
	    &= \dfrac{(1)(rp^{-2}(r+1-p)) - (rp^{-1})^2}{(1)^2} \\
	    &= rp^{-2}(r+1-p) - (rp^{-1})^2 \\
	    &= r^2p^{-2} + rp^{-2} - rp^{-1} - r^2p^{-2} \\
	    &= rp^{-2} - rp^{-1} \\
	    &= rp^{-2}(1 - p) \\
	    &= \dfrac{r(1-p)}{p^2}
	  \end{align*}


	\end{enumerate}

      %7
      \item
	Start by writing out $E(X^r)$ at $r=1$ and $r=2$:
	\begin{align*}
	 E(X^1) = \mathlarger{\sum}_{x\in S}{x\cdot f(x)} = 5^1 \\
	 E(X^2) = \mathlarger{\sum}_{x\in S}{x^2 \cdot f(x)} = 5^2 
	\end{align*}
	
	This tells us that the probability distribution has a mean of 5 with a variance of 0. So we can
	infer that the pmf is $f(x) = 1$ for $x=5$. 
	
	The mgf is then $M(t) = E(e^{Xt}) = \mathlarger{\sum}_{x=5}{e^{xt}f(x)} = e^{5t}$.

      \addtocounter{enumi}{1}
      
      %9
      \item
	$1 + \dfrac{4}{3} + \dfrac{4}{2} + \dfrac{4}{1} = \dfrac{25}{3} = 8.333$
    \end{enumerate}
    
    \textbf{Section 2.6}
    \begin{enumerate}
     %1
     \item 
      \begin{enumerate}
       %a
       \item 
	$P(2 \le X \le 5;\lambda=4) = P(X \le 5;\lambda=4) - P(X \le 1;\lambda=4) = 0.785 - 0.092 = 0.693$
       %b
       \item
	$P(X \ge 3;\lambda=4) = 1 - P(X \le 2;\lambda=4) = 1 - 0.238 = 0.762$
       
       %c
       \item
	$P(X \le 3;\lambda=4) = 0.433$	
      \end{enumerate}
     
     %2
     \item
      $P(X=2;\lambda=2) = \dfrac{2^2e^{-2}}{2!} = \dfrac{4}{2e^2} = 0.2707$
     
     %3
     \item
      $P(X > 10; \lambda=11) = 1 - P(X \le 10; \lambda=11) = 1 - 0.460 = 0.540$
     
     %4
     \item
      \begin{align*}
       3P(X=1) &= P(X=2) \\
       3\left(\dfrac{\lambda^1 e^{-1}}{1!}\right) &= \dfrac{\lambda^2 e^{-2}}{2!} \\
       3\left(\dfrac{\lambda}{e}\right) &= \dfrac{\lambda^2}{2e} \\
       3\left(\dfrac{2e}{e}\right) &= \dfrac{\lambda^2}{\lambda} \\
       6 &= \lambda
      \end{align*}

      Therefore, $P(X=4;\lambda=6) = \dfrac{6^4e^{-6}}{4!} = 0.1339$
     
     %5
     \item
      $P(X \le 1;\lambda=1.5) = 0.558$
     
     %6
     \item
      $P(X = 0;\lambda=0.5) = 0.607$
     \addtocounter{enumi}{1}
     
     %8
     \item
      \begin{enumerate}
       %a
       \item
	$P(X \le 1;\lambda=5) = 0.040$
       
       %b
       \item
	$P(4 \le X \le 6;\lambda=5) = P(X \le 6; \lambda=5) - P(X \le 3;\lambda=5) = 0.762 - 0.265 = 0.497$
      \end{enumerate}
     
     %9
     \item
      \begin{enumerate}
       %a
       \item
	We are given that random variable X, the number of requests per day, follows a Poisson distribution
	with $\lambda=3$. Therefore, we can define random variable Y, the number of newspapers sold per day,
	as follows:
	
	\begin{center}	 
	  \begin{tabular}{c | c}
		Y & $g(y)$ \\
		\hline
		0 & $f(X=0) = 0.0498$ \\
		1 & $f(X=1) = 0.1494$ \\
		2 & $f(X=2) = 0.2240$ \\
		3 & $f(X=3) = 0.2240$ \\
		4 & $f(X \ge 4) = 1 - f(X \le 3) = 0.353$
	  \end{tabular} 
	\end{center}
	
	Therefore, the expected value of Y, the number sold, is $$E(Y) = 0(0.0498) + 1(0.1494) + 2(0.2240) 
	 + 3(0.2240) + 4(0.353) = 2.6814$$
       
       %b
       \item
	We are looking for n, such that $P(X > n) < 0.05$. We can rewrite that as 
	\begin{align*}
	  P(X > n) &< 0.05 \\
	  1 - P(X \le n) &< 0.05 \\
	  -P(X \le n) &< -0.95 \\
	  P(X \le n) &> 0.95
	\end{align*}
	
	By looking at the table in Appendix B, we can see that for $\lambda=3$, the first $x$ value 
	where $P(X \le x) > 0.95$ is when $x=6$. 
      \end{enumerate}
     
     %10
     \item
      If X is a Poisson random variable with $\mu = 9$, that means that $\mu=\sigma^2=\lambda=9$, $\sigma=3$.
      \begin{align*}
       P(\mu - 2\sigma < X < \mu + 2\sigma; \lambda=9) &= P(9-2(3) < X < 9 + 2(3); \lambda=9)\\
	&= P(3 < X < 15; \lambda=9) \\
	&= P(X \le 14;\lambda=9) - P(X \le 3;\lambda=9) \\
	&= 0.959 - 0.021 \\
	&= 0.938
      \end{align*}

    \end{enumerate}

\end{document}