\documentclass{article}
    
    
    \usepackage{graphicx} % Used to insert images
    \usepackage{adjustbox} % Used to constrain images to a maximum size 
    \usepackage{color} % Allow colors to be defined
    \usepackage{enumerate} % Needed for markdown enumerations to work
    \usepackage{geometry} % Used to adjust the document margins
    \usepackage{amsmath} % Equations
    \usepackage{amssymb} % Equations
    \usepackage{eurosym} % defines \euro
    \usepackage[mathletters]{ucs} % Extended unicode (utf-8) support
    \usepackage[utf8x]{inputenc} % Allow utf-8 characters in the tex document
    \usepackage{fancyvrb} % verbatim replacement that allows latex
    \usepackage{grffile} % extends the file name processing of package graphics 
                         % to support a larger range 
    % The hyperref package gives us a pdf with properly built
    % internal navigation ('pdf bookmarks' for the table of contents,
    % internal cross-reference links, web links for URLs, etc.)
    \usepackage{hyperref}
    \usepackage{longtable} % longtable support required by pandoc >1.10
    \usepackage{booktabs}  % table support for pandoc > 1.12.2
    \usepackage{indentfirst}
    \usepackage{floatrow}
    
    \newcommand\perm[2]{{}_{#1}P_{#2}}%
    
    
    \title{Homework 3}
    \author{Roly Vicar\'ia \\ STAT414 Spring 2016}
    
\begin{document}
    
    \maketitle
    
    Section 1.4
    \begin{enumerate}
     
     \addtocounter{enumi}{1}
     
     %2
     \item 
      \begin{enumerate}
       \item $P(A \cup B) = P(A) + P(B) - P(A \cap B) = 0.3 + 0.6 - 0.18 = 0.72$
       \item $P(A | B) = \dfrac{P(A \cap B)}{P(B)} = \dfrac{0}{P(B)} = 0$
      \end{enumerate}

     \addtocounter{enumi}{2}
     
     %5
     \item 
      $P(A) = .8$ \\
      $P(B) = .5$ \\
      $P(A \cup B) = .9$ \\
      \\
      $P(A \cup B) = P(A) + P(B) - P(A \cap B)$ \\
      $.9 = .8 + .5 - P(A \cap B)$ \\
      $P(A \cap B) = .4 = .8 \cdot .5 = P(A)\cdot P(B)$\\
      Therefore, A and B are independent events.

      %6
      \item 
	$A$ and $(B \cap C)$: \\
	\begin{align*}
	 P(A \cap (B \cap C)) &= P(A \cap B \cap C) \\
	  &= P(A)P(B)P(C)
	\end{align*}
	
	
	$A$ and $(B \cup C)$: \\
	\begin{align*}
	 P(A \cap (B \cup C)) &= P((A \cap B) \cup (A \cap C)) \\
	  &= P(A \cap B) + P(A \cap C) - P(A \cap B \cap C) \\
	  &= P(A)P(B) + P(A)P(C) - P(A)P(B)P(C) \\
	  &= P(A)[P(B) + P(C) - P(B)P(C)] \\
	  &= P(A)[P(B) + P(C) - P(B \cap C)] \\
	  &= P(A)P(B \cup C)
	\end{align*}

	\newpage
	$A'$ and $(B \cap C')$: \\
	\begin{align*}
	 P(A' \cap (B \cap C')) &= 1 - P(A \cup (B' \cup C)) \\
	  &= 1 - [P(A) + P(B') + P(C) - P(A \cap B') - P(A \cap C) - P(B' \cap C) + P(A \cap B' \cap C)] \\
	  &= 1 - P(A) - P(B) - P(C) + P(A \cap B') + P(A \cap C) + P(B' \cap C) - P(A \cap B'\cap C) \\
	  &= [1 - P(A)][1 - P(B')][1 - P(C)] \\
	  &= P(A)P(B')P(C)
	\end{align*}

	
	$A'$, $B'$, and $C'$ are mutually independent: \\
	Condition (a): 
	\begin{align*}
	 P(A' \cap B' \cap C') &= 1 - P(A \cup B \cup C) \\
	  &= 1 - [P(A) + P(B) + P(C) - P(A \cap B) - P(A \cap C) - P(B \cap C) + P(A \cap B \cap C)] \\
	  &= 1 - [P(A) + P(B) + P(C) - P(A)P(B) - P(A)P(C) - P(B)P(C) + P(A)P(B)P(C)] \\
	  &= 1 - P(A) - P(B) - P(C) + P(A)P(B) + P(A)P(C) + P(B)P(C) - P(A)P(B)P(C) \\
	  &= [1- P(A)][1 - P(B)][1 - P(C)] \\
	  &= P(A')P(B')P(C')
	\end{align*}

	Condition (b):
	\begin{align*}
	 P(A' \cap B') &= 1 - P(A \cup B) \\
	  &= 1 - [P(A) + P(B) - P(A \cap B)] \\
	  &= 1 - P(A) - P(B) + P(A)P(B) \\
	  &= [1 - P(A)][1 - P(B)] \\
	  &= P(A')P(B')
	\end{align*}
	\begin{align*}
	 P(A' \cap C') &= 1 - P(A \cup C) \\
	  &= 1 - [P(A) + P(C) - P(A \cap C)] \\
	  &= 1 - P(A) - P(C) + P(A)P(C) \\
	  &= [1 - P(A)][1 - P(C)] \\
	  &= P(A')P(C')
	\end{align*}
	\begin{align*}
	 P(B' \cap C') &= 1 - P(B \cup C) \\
	  &= 1 - [P(B) + P(C) - P(B \cap C)] \\
	  &= 1 - P(B) - P(C) + P(B \cap C) \\
	  &= [1 - P(B)][1 - P(C)] \\
	  &= P(B')P(C')
	\end{align*}

      %7
      \item
	\begin{enumerate}
	 \item 
	    $P(A_1 \cap A_2' \cap A_3') = P(A_1)\cdot P(A_2')\cdot P(A_3') = .5\cdot.3\cdot.4 = .06$\\
	    $P(A_1' \cap A_2 \cap A_3') = P(A_1')\cdot P(A_2)\cdot P(A_3') = .5\cdot.7\cdot.4 = .14$\\
	    $P(A_1' \cap A_2' \cap A_3) = P(A_1')\cdot P(A_2')\cdot P(A_3) = .5\cdot.3\cdot.6 = .09$\\
	    $P(exactly\ one\ player\ successful) = .06+.14+.09 = 0.29$
	 \item 
	    $P(A_1' \cap A_2 \cap A_3) = P(A_1')\cdot P(A_2)\cdot P(A_3) = .5\cdot.7\cdot.6 = .21$\\
	    $P(A_1 \cap A_2' \cap A_3) = P(A_1)\cdot P(A_2')\cdot P(A_3) = .5\cdot.3\cdot.6 = .09$\\
	    $P(A_1 \cap A_2 \cap A_3') = P(A_1)\cdot P(A_2)\cdot P(A_3') = .5\cdot.7\cdot.4 = .14$\\
	    $P(exactly\ one\ player\ misses) = .21+.09+.14 = 0.44$
	\end{enumerate}
      \addtocounter{enumi}{3}
      
      %11
      \item 
	\begin{enumerate}
	 \item 
	    If events $A$ and $B$ are mutually exclusive, then by definition, that means that 
	 their intersection, $A\cap B$, is always 0. Therefore $P(A|B) = \dfrac{P(A\cap B)}{P(B)} = 0$. 
	 This may or may not be equal to $P(A)$. So, they can be independent if $P(A) = 0$ or 
	 $P(B) = 0$, but not always.
	 
	 \item 
	    We have two cases: 
	      \begin{enumerate}
	       \item $P(B|A) = \dfrac{P(A \cap B)}{P(A)} = \dfrac{P(A)}{P(A)} = 1$. This
		means that if $P(B) = 1$, then they can be independent. 
	    
	       \item $P(A|B) = \dfrac{P(A \cap B)}{P(B)} = \dfrac{P(A)}{P(B)}$. This can equal $P(A)$
		if $P(A) = 0$ or if $P(B) = 1$.
	      \end{enumerate}	    
	\end{enumerate}
	  
      %12
      \item 
	\begin{enumerate}
	 \item $P(HHTHT) = .5\cdot.5\cdot.5\cdot.5\cdot.5 = (.5)^5 = 0.03125$
	 \item $P(THHHT) = (.5)^5 = 0.03125$
	 \item $P(HTHTH) = (.5)^5 = 0.03125$
	 \item $P(three\ heads\ occurring) 
	 = \dbinom{5}{3}\cdot\left(\dfrac{1}{2}\right)^3\cdot\left(\dfrac{1}{2}\right)^2 = 0.3125$
	\end{enumerate}
      \addtocounter{enumi}{5}

      %18
      \item 
	\begin{enumerate}
	 \item 7
	 \item $\left(\dfrac{1}{2}\right)^7 = \dfrac{1}{128} = 0.0078125$
	 \item $32 + 16 + 8 + 4 + 2 + 1 = 63$
	 \item $\left(\dfrac{1}{2}\right)^{63} = 0.00000000000000000010842022$ \\
	 I think the statement is wrong and actually represents a higher probability than calculated
	 above. I think it's closer to ``1 in 9,223,372,036,854,775,808''
	\end{enumerate}	
    \end{enumerate}
    
    Section 1.5
    \begin{enumerate}
     \addtocounter{enumi}{1}
     
     %2
     \item 
      \begin{enumerate}
       \item $P(G) = P((G \cap A) \cup (G \cap B)) = P(G|A)P(A) + P(G|B)P(B) 
       = .85\cdot.4 + .75\cdot.6 = 0.79$
       \item $P(A|G) = \dfrac{P(G|A)P(A)}{P(G)} = \dfrac{.85\cdot.4}{.79} = .4304$
      \end{enumerate}
      
     %3
     \item 
	The percentage of patients with regular heartbeat and low blood pressure is 15.1\%
	\begin{longtable}[c]{@{}lcccl@{}}
	  \toprule
	  & Low & Normal & High\tabularnewline
	  \midrule
	  \endhead
	  irregular & 39 & 71.5 & 59.5 & 170\tabularnewline
	  regular & 151 & & & 830\tabularnewline
	  & 190 & 650 & 160 & 1000 \tabularnewline
	  \bottomrule
	\end{longtable}
	
     %4
     \item 
	Y = event that driver is in the 16-25 age group \\
	A = event that driver has at least 1 accident \\
	\\
	$P(Y|A) = \dfrac{P(A|Y)P(Y)}{P(A)} = \dfrac{.05\cdot.1}{.05\cdot.1 + .02\cdot.55 + 
	  .03\cdot.2 + .04\cdot.15} = 0.1786$
     \addtocounter{enumi}{3}
      
     %8
     \item 
	Total probability of $R$:
	\begin{align*}
	  P(R) &= P(R|B_1)P(B_1)+P(R|B_2)P(B_2)+P(R|B_3)P(B_3)+P(R|B_4)P(B_4) \\
	       &= .1\cdot.4 + .05\cdot.3 + .03\cdot.2 + .02\cdot.1 \\
	       &= 0.063
	\end{align*}
	\newline
	
	Posterior probabilities:\\
	$P(B_1|R) = \dfrac{P(R|B_1)P(B_1)}{P(R)} = \dfrac{.1\cdot.4}{.063} = 0.6349$\\
	$P(B_2|R) = \dfrac{P(R|B_2)P(B_2)}{P(R)} = \dfrac{.05\cdot.3}{.063} = 0.2381$\\
	$P(B_3|R) = \dfrac{P(R|B_3)P(B_3)}{P(R)} = \dfrac{.03\cdot.2}{.063} = 0.09524$\\
	$P(B_4|R) = \dfrac{P(R|B_4)P(B_4)}{P(R)} = \dfrac{.02\cdot.1}{.063} = 0.03174$
    \end{enumerate}
\end{document}
