\documentclass{article}
    
    \usepackage{graphicx} % Used to insert images
    \usepackage{adjustbox} % Used to constrain images to a maximum size 
    \usepackage{color} % Allow colors to be defined
    \usepackage{enumerate} % Needed for markdown enumerations to work
    \usepackage{geometry} % Used to adjust the document margins
    \usepackage{amsmath} % Equations
    \usepackage{amssymb} % Equations
    \usepackage{eurosym} % defines \euro
    \usepackage[mathletters]{ucs} % Extended unicode (utf-8) support
    \usepackage[utf8x]{inputenc} % Allow utf-8 characters in the tex document
    \usepackage{fancyvrb} % verbatim replacement that allows latex
    \usepackage{grffile} % extends the file name processing of package graphics 
                         % to support a larger range 
    % The hyperref package gives us a pdf with properly built
    % internal navigation ('pdf bookmarks' for the table of contents,
    % internal cross-reference links, web links for URLs, etc.)
    \usepackage{hyperref}
    \usepackage{longtable} % longtable support required by pandoc >1.10
    \usepackage{booktabs}  % table support for pandoc > 1.12.2
    \usepackage{indentfirst}
    \usepackage{floatrow}
    \usepackage{relsize}
    \usepackage{multirow}
        
    \newcommand\perm[2]{{}_{#1}P_{#2}}%
    \newcommand\todo[1]{\textbf{TODO: #1}}% 
    \newcommand\numberthis{\addtocounter{equation}{1}\tag{\theequation}}
    \newcommand\seteq{\mathrel{\overset{\makebox[0pt]{\mbox{\normalfont\small\sffamily set}}}{=}}}
    \newcommand\mfrac[2]{\left(\dfrac{#1}{#2}\right)}
    \newcommand\lint{\mathlarger{\int}}
    \newcommand\lsum{\mathlarger{\sum}}
    \newcommand\lprod{\mathlarger{\prod}}
    \newcommand\myskip[1]{\addtocounter{enumi}{#1}}
    
    \title{Homework 13}
    \author{Roly Vicar\'ia \\ STAT414 Spring 2016}
    
\begin{document}
    
    \maketitle
    
    \textbf{Section 5.3}
    \begin{enumerate}
     %1
     \item
      \begin{enumerate}
       %a
       \item
	$P(X_1 = 3, X_2 = 5) = P(X_1 = 3)P(X_2 = 5) = \mfrac{2^3e^{-2}}{3!}\mfrac{3^5e^{-3}}{5!}
	  = 0.01819$
       
       %b
       \item
	$P(X_1 + X_2 = 1) = P(X_1=1, X_2=0) + P(X_1=0, X_2=1) = \\
	  P(X_1=1)P(X_2=0) + P(X_1=0)P(X_2=1) = \mfrac{2^1e^{-2}}{1!}\mfrac{3^0e^{-3}}{0!} 
	  + \mfrac{2^0e^{-2}}{0!}\mfrac{3^1e^{-3}}{1!} \\
	  = 0.03369$
      \end{enumerate}
     \myskip{1}
     
     %3
     \item
      \begin{enumerate}
       %a
       \item
	$P(0.5 < X_1 < 1 \text{ and } 0.4 < X_2 < 0.8) = P(0.5 < X_1 < 1)P(0.4 < X_2 < 0.8) \\
	= \left[\lint_{0.5}^1 { 2x_1 dx_1 }\right]\left[\lint_{0.4}^{0.8} {4x_2^3 dx_2}\right] 
	= x_1^2 \Big|_{0.5}^1 \cdot x_2^4 \Big|_{0.4}^{0.8} = (1 - 0.25)(0.4096 - 0.0256)
	= 0.288$ 
	
       %b
       \item
	$E(X_1^2X_2^3) = E(X_1^2)E(X_2^3) = \left[\lint_0^1 {x_1^2 2x_1 dx}\right]
	\left[\lint_0^1 {x_2^3 4x_2^3 dx}\right] = \dfrac{x_1^4}{2} \Big|_0^1 \cdot 
	\dfrac{4x_2^7}{7} \Big|_0^1 = \dfrac{1}{2}\cdot \dfrac{4}{7} = \dfrac{2}{7}$
      \end{enumerate}
     \myskip{1}
     
     %5
     \item
      pmf of $Y$:
      \begin{center}
	\begin{tabular}{c | c | c | c | c | c}
	  $y = x_1 + x_2$ & 2 & 3 & 4 & 5 & 6 \\
	  \hline	    
	  $g(y)$ & 1/36 & 4/36 & 10/36 & 12/36 & 9/36 \\ 
	\end{tabular}
      \end{center}
      
      Mean and variance \#1:\\
      $E(Y) = \lsum_y {y g(y)} = 2\mfrac{1}{36} + 3\mfrac{4}{36} + 4\mfrac{10}{36} + 5\mfrac{12}{36}
      + 6\mfrac{9}{36} = \dfrac{14}{3}$ \\
      $E(Y^2) = \lsum_y {y^2 g(y)} = 4\mfrac{1}{36} + 9\mfrac{4}{36} + 16\mfrac{10}{36} + 25\mfrac{12}{36}
      + 36\mfrac{9}{36} = \dfrac{206}{9}$ \\
      $Var(Y) = E(Y^2) - [E(Y)]^2 = \dfrac{10}{9}$ \\
      
      Mean and variance \#2: \\
      $E(Y) = E(X_1 + X_2) = E(X_1) + E(X_2) = 2E(X)
	= 2\left[1\mfrac{1}{6} + 2\mfrac{2}{6} + 3\mfrac{3}{6}\right] \\ 
	= 2\mfrac{14}{6} = \dfrac{14}{3}$ \\
      $Var(Y) = Var(X_1) + Var(X_2) = 2Var(X) = 2\left[\lsum_x{(x-14/6)^2 f(x)}\right] \\
	= 2\left[\left(1-\dfrac{14}{6}\right)^2\mfrac{1}{6} 
	  + \left(2 - \dfrac{14}{6}\right)^2\mfrac{2}{6}
	  + \left(3 - \dfrac{14}{6}\right)^2\mfrac{3}{6}\right] = \dfrac{10}{9}$
     
     %6
     \item
      The mean of $X_1$ and $X_2$,
      \begin{align*}
       E(X_1) = E(X_2) = E(X) = \lint_0^1{x \cdot 6x(1-x) dx} = \dfrac{1}{2}
      \end{align*}
      
      Therefore,
      \begin{align*}
       E(Y) = E(X_1 + X_2) = E(X_1) + E(X_2) = 2E(X) = 2(1) = 2
      \end{align*}

      The variance of $X_1$ and $X_2$,
      \begin{align*}
       Var(X_1) = Var(X_2) = Var(X) = \lint_0^1{x^2 \cdot 6x(1-x) dx} = \dfrac{6}{20}
      \end{align*}

      Therefore,
      \begin{align*}
       Var(Y) = Var(X_1 + X_2) = Var(X_1) + Var(X_2) = 2Var(X) = 2\mfrac{6}{20} = \dfrac{6}{10}
      \end{align*}
     \myskip{4}
     
     %11
     \item
      \begin{enumerate}
       %a
       \item 	
	\begin{align*}
	  P(X_1=2, X_2=2, X_3=5) &= P(X_1 = 2)P(X_2 = 2)P(X_3 = 5) \\
	    &= \left[\dbinom{4}{2}\mfrac{1}{2}^2\mfrac{1}{2}^2\right]
	      \left[\dbinom{6}{2}\mfrac{1}{3}^2\mfrac{2}{3}^4\right]
	      \left[\dbinom{12}{5}\mfrac{1}{6}^5\mfrac{5}{6}^7\right] \\
	    &= 0.00351
	\end{align*}
       
       %b
       \item
	\begin{align*}
	  E(X_1 X_2 X_3) &= E(X_1)E(X_2)E(X_3) \\
	    &= 4\mfrac{1}{2} \cdot 6\mfrac{1}{3} \cdot 12\mfrac{1}{6} \\
	    &= 8
	\end{align*}
       %c
       \item
	\begin{align*}
	  E(Y) &= E(X_1 + X_2 + X_3) \\
	    &= E(X_1) + E(X_2) + E(X_3) \\
	    &= 3(2) \\
	    &= 6
	\end{align*}
	
	\begin{align*}
	  Var(Y) &= Var(X_1 + X_2 + X_3) \\
	    &= Var(X_1) + Var(X_2) + Var(X_3) \\
	    &= 4\mfrac{1}{2}\mfrac{1}{2} + 6\mfrac{1}{3}\mfrac{2}{3} + 12\mfrac{1}{6}\mfrac{5}{6} \\
	    &= 4
	\end{align*}
      \end{enumerate}
     
     %12
     \item
      
      \begin{align*}
       P(1 < \text{min}X_i) &= P(1 < X_1, 1 < X_2, 1 < X_3) \\
	&= P(1 < X_1)P(1 < X_2)P(1 < X_3) \\
	&= [1 - P(X_1 \le 1)][1 - P(X_2 \le 1)][1 - P(X_3 \le 1)] \\
	&= (e^{-1})^3 \\
	&= e^{-3} \approx 0.04979
      \end{align*}
     \myskip{4}
     
     %17
     \item
      We're given that,
      $$\sigma_X^2 = 8100,\ \ \sigma_Y^2 = 10000,\ \ \sigma_{X+Y}^2 = 20000$$
      
      Since, $Var(X+Y)$ is not equal to the sum of $Var(X)$ and $Var(Y)$, we know they must be 
      correlated. We can find the correlation coefficient, by solving the following for $rho_{XY}$,
      $$\sigma_{X+Y}^2 = \sigma_X^2 + \sigma_Y^2 + \rho_{XY} \sigma_X \sigma_Y$$
      $$\rho_{XY} = \dfrac{20000-8100-10000}{90(100)} = 0.2111$$
      
      We can define $W = X+500$, and $T = 1.08Y$, which are linear combinations of $X$ and $Y$, respectively.
      Thus, to compute, $Var(W)$ and $Var(T)$,
      $$Var(W) = Var(X+500) = Var(X) = 8100$$
      $$Var(T) = Var(1.08Y) = 1.08^2 Var(Y) = 11664$$
      
      Therefore, we can define $Z = W+T$, and compute its variance as follows,
      \begin{align*}
       Var(Z) &= Var(W) + Var(T) + \rho_{XY} \sigma_W \sigma_T \\
	&= 8100 + 11664 + 0.2111(90)(108) \\
	&= 21816
      \end{align*}
     
     %18
     \item
      Since these are random samples of a gamma distribution with $\alpha = \theta = 2$, then
      $$E(X_1) = E(X_2) = E(X_3) = E(X) = \alpha \theta$$
      and,
      $$Var(X_1) = Var(X_2) = Var(X_3) = Var(X) = \alpha \theta^2$$
      
      Therefore,
      \begin{align*}
	E(X_1+X_2+X_3) &= E(X_1) + E(X_2) + E(X_3) \\
	  &= 3(\alpha \theta) \\
	  &= 3(2)(2) \\
	  &= 12
      \end{align*}
      \begin{align*}
       Var(X_1 + X_2 + X_3) &= Var(X_1) + Var(X_2) + Var(X_3) \\
	&= 3(\alpha \theta^2) \\
	&= 3(2)(2)^2 \\
	&= 24
      \end{align*}
    \end{enumerate}
    
    \textbf{Section 5.4}
    \begin{enumerate}
     \myskip{1}
     %2
     \item 
      Binomial mgf: $(q + pe^t)^n$
      
      We are given that $Y = X_1 + X_2$, where $X_1 \sim b(n_1, p)$ and $X_2 \sim b(n_2, p)$.\\
      Therefore, $M_{X_1} = (q + pe^t)^{n_1}$ and $M_{X_2} = (q + pe^t)^{n_2}$. 
      
      The mgf of $Y$ is 
      \begin{align*}
       M_Y(t) &= M_{X_1}(t)M_{X_2}(t) \\
	&= (q + pe^t)^{n_1}(q + pe^t)^{n_2} \\
	&= (q + pe^t)^{(n_1 + n_2)}
      \end{align*}
      
      This last line is the mgf of a binomial distribution, $b(n_1+n_2, p)$.
     
     %3
     \item
      \begin{enumerate}
       %a
       \item 
	$M_{X_1}(t) = e^{2(e^t - 1)}$,  $M_{X_2}(t) = e^{(e^t - 1)}$,  $M_{X_3}(t) = e^{4(e^t - 1)}$ \\
	Therefore, 
	\begin{align*}
	M_Y(t) &= M_{X_1}(t)M_{X_2}(t)M_{X_3}(t) \\
	  &= e^{2(e^t - 1)} e^{(e^t - 1)} e^{4(e^t - 1)} \\
	  &= e^{7(e^t - 1)}
	\end{align*}
	
       %b
       \item
	$Y$ has a Poisson distribution with mean 7.
       
       %c
       \item
	$P(3 \le Y \le 9) = P(Y \le 9) - P(Y \le 2) = 0.830 - 0.030 = 0.800$
      \end{enumerate}
     
     %4
     \item
      If we have a random variable, $Y$, equal to the sum of $n$ Poisson random variables with 
      means $\mu_1, \mu_2, \cdots, \mu_n$, with each having mgf, 
      $$M_{X_i}(t) = e^{\mu_i(e^t - 1)}$$
      then by theorem 5.4-1, the moment-generating function,
      \begin{align*}
       M_Y(t) &= \lprod_{i=1}^n M_{X_i}(t) \\
	&= (e^{\mu_1(e^t - 1)})(e^{\mu_2(e^t - 1)})\cdots(e^{\mu_n(e^t - 1)}) \\
	&= e^{(e^t - 1)\lsum_{i=1}^n {\mu_i}}
      \end{align*}
      
      This last line is the mgf of a Poisson random variable with mean $\lsum_{i=1}^n {\mu_i}$
     
     %5
     \item
      By Corollary 5.4-2, $W$ has a distribution that is $\chi^2(7)$. Therefore,
      \begin{align*}
       P(1.69 < W < 14.07) &= P(W < 14.07) - P(W < 1.69) \\
	&= 0.95 - 0.025 = 0.925
      \end{align*}
     \myskip{2}
     
     %8
     \item
      We have $h$ mutually independent and identically exponential random variables with mean $\theta$.
      Each of these $h$ random variables has mgf $M_{X_i}(t) = (1-\theta)^{-1}$. By theorem, 5.4-1,
      the moment-generating function of $W$ is,
      \begin{align*}
       M_W(t) &= \lprod_{i=1}^h M_{X_i}(t) \\
	&= \lprod_{i=1}^h {(1 - \theta)^{-1}} \\
	&= ((1 - \theta)^{-1})^h \\
	&= (1 - \theta)^{-h}
      \end{align*}
      
      This last line is the mgf of a Gamma distribution with parameters $\alpha = h$ and $\theta$.
     \myskip{5}
     
     %14
     \item
      $Y = X_1 + X_2 + X_3 \sim Poisson(6)$ \\
      $P(Y = 7) = \dfrac{6^7 e^{-6}}{7!} = 0.1377$
     \myskip{1}
     
     %16
     \item
      $Y = X_1 + X_2 + X_3 + X_4 \sim Poission(8)$ \\
      $P(Y > 10) = 1 - P(Y \le 10) = 1 - 0.816 = 0.184$
     \myskip{2}
     
     %19
     \item
      The sum of the three exponential random variables equates to a gamma random variable
      with $\alpha = 3$ and $\theta = 2$. Therefore,
      \begin{align*}
       P(X \le 6) &= \lint_0^6 {\dfrac{1}{\Gamma(3)2^3} x^2 e^{-x/2} dx} \\
	&= \dfrac{1}{\Gamma(3)2^3} \lint_0^6 {x^2 e^{-x/2} dx} 
      \end{align*}
      
      Doing integration by parts with $u = x^2$ and $dv = e^{-x/2} dx$. Therefore, $du = 2x dx$
      and $v = -2e^{-x/2}$
      
      \begin{align*}
	\dfrac{1}{\Gamma(2)2^3} \lint_0^6 {x^2 e^{-x/2} dx} &= \dfrac{1}{\Gamma(3)2^3}
	    \left[-2x^2e^{-x/2} \Big|_0^6 - \lint_0^6 {-4xe^{-x/2} dx}\right] \\
	  &= \dfrac{1}{\Gamma(3)2^3} \left[-72e^{-3} + 4\lint_0^6 {xe^{-x/2}dx} \right] 
      \end{align*}
      
      Again doing integration by parts with $u = x$ and $dv = e^{-x/2} dx$. Therefore, $du = dx$
      and $v = -2e^{-x/2}$
      
      \begin{align*}
       \dfrac{1}{\Gamma(3)2^3} \left[-72e^{-3} + 4\lint_0^6 {xe^{-x/2}dx} \right] &= 
	    \dfrac{1}{\Gamma(3)2^3}\left\{-72e^{-3} + 4 \left[-2xe^{-x/2} \Big|_0^6 - 
	      \lint_0^6 {-2e^{-x/2}dx}\right]\right\} \\
	  &= \dfrac{1}{\Gamma(3)2^3}\left\{-72e^{-3} + 4 \left[-12e^{-3} + 2 
	      \lint_0^6 {e^{-x/2}dx}\right]\right\} \\
	  &= \dfrac{1}{\Gamma(3)2^3} \left[-120e^{-3} + 8 \lint_0^6 {e^{-x/2}dx}\right] \\
	  &= \dfrac{1}{\Gamma(3)2^3} \left[-120e^{-3} - 16(e^{-x/2} \Big|_0^6)\right] \\
	  &= \dfrac{1}{\Gamma(3)2^3} \left[-120e^{-3} - 16 (e^{-3} - 1)\right] \\
	  &= \dfrac{1}{16} \left(-136e^{-3} + 16\right) \\
	  &= -8.5e^{-3} + 1 \approx 0.5768
      \end{align*}



    \end{enumerate}

\end{document}