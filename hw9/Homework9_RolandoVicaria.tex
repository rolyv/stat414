\documentclass{article}
    
    \usepackage{graphicx} % Used to insert images
    \usepackage{adjustbox} % Used to constrain images to a maximum size 
    \usepackage{color} % Allow colors to be defined
    \usepackage{enumerate} % Needed for markdown enumerations to work
    \usepackage{geometry} % Used to adjust the document margins
    \usepackage{amsmath} % Equations
    \usepackage{amssymb} % Equations
    \usepackage{eurosym} % defines \euro
    \usepackage[mathletters]{ucs} % Extended unicode (utf-8) support
    \usepackage[utf8x]{inputenc} % Allow utf-8 characters in the tex document
    \usepackage{fancyvrb} % verbatim replacement that allows latex
    \usepackage{grffile} % extends the file name processing of package graphics 
                         % to support a larger range 
    % The hyperref package gives us a pdf with properly built
    % internal navigation ('pdf bookmarks' for the table of contents,
    % internal cross-reference links, web links for URLs, etc.)
    \usepackage{hyperref}
    \usepackage{longtable} % longtable support required by pandoc >1.10
    \usepackage{booktabs}  % table support for pandoc > 1.12.2
    \usepackage{indentfirst}
    \usepackage{floatrow}
    \usepackage{relsize}
    
    \newcommand\perm[2]{{}_{#1}P_{#2}}%
    \newcommand\todo[1]{\textbf{TODO: #1}}% 
    \newcommand\numberthis{\addtocounter{equation}{1}\tag{\theequation}}
    \newcommand\seteq{\mathrel{\overset{\makebox[0pt]{\mbox{\normalfont\small\sffamily set}}}{=}}}
    
    \title{Homework 9}
    \author{Roly Vicar\'ia \\ STAT414 Spring 2016}
    
\begin{document}
    
    \maketitle
    
    \textbf{Section 3.3}
    \begin{enumerate}
     %1
     \item 
      \begin{enumerate}
       %a
       \item 
	$P(0.53 < Z \le 2.06) = \Phi(2.06) - \Phi(0.53) = 0.9803 - 0.7019 = 0.2784$
       
       %b
       \item 
	$P(-0.79 \le Z < 1.52) = \Phi(1.52) - \Phi(-0.79) = 0.9357 - 0.2148 = 0.7209$
       
       %c
       \item
	$P(Z > -1.77) = \Phi(1.77) = 0.9616$
       
       %d
       \item
	$P(Z > 2.89) = 0.0019$
       
       %e
       \item
	$P(|Z| < 1.96) = P(-1.96 < Z < 1.96) = \Phi(1.96) - P(Z > 1.96) = 0.9750 - 0.0250 = 0.95$
       
       %f
       \item
	$P(|Z| < 1) = P(-1 < Z < 1) = \Phi(1) - P(Z > 1) = 0.8413 - 0.1587 = 0.6826$
       
       %g
       \item
	$P(|Z| < 2) = P(-2 < Z < 2) = \Phi(2) - P(Z > 2) = 0.9772 - 0.0228 = 0.9544$
       
       %h
       \item
	$P(|Z| < 3) = P(-3 < Z < 3) = \Phi(3) - P(Z > 3) = 0.9987 - 0.0013 = 0.9974$
      \end{enumerate}
     \addtocounter{enumi}{1}
     
     %3
     \item
      \begin{enumerate}
       %a
       \item
	$P(Z \ge c) = 0.025$ \\
	$c = 1.96$
       
       %b
       \item
	$P(|Z| \le c) = 0.95$ \\
	$c = 1.96$
       
       %c
       \item
	$P(Z > c) = 0.05)$ \\
	$c = 1.645$
       
       %d
       \item
	$P(|Z| \le c) = 0.90$ \\
	$c = 1.645$
      \end{enumerate}
     
     %4
     \item
      \begin{enumerate}
       %a
       \item
	$z_{0.10} = 1.282$
       
       %b
       \item
	$-z_{0.05} = -1.645$
       
       %c
       \item
	$-z_{0.0485} = -1.66$
       
       %d
       \item
	$z_{0.9656} = -1.82$
      \end{enumerate}
     
     %5
     \item
      $X \sim N(6,25)$
      \begin{enumerate}
       %a
       \item 
	$P(6 \le X \le 12) = P(0 \le Z \le 1.2) = \Phi(1.2) - \Phi(0) = 0.8849 - 0.5 = 0.3849$
       
       %b
       \item 
	$P(0 \le X \le 8) = P(-1.2 \le Z \le 0.4) = \Phi(0.4) - \Phi(-1.2) 
	    = 0.6554 - 0.1151 = 0.5403$
       
       %c
       \item 
	$P(-2 < X \le 0) = P(-1.6 < Z \le -1.2) = \Phi(-1.2) - \Phi(-1.6) 
	    = 0.1151 - 0.0548 = 0.0603$
       
       %d
       \item 
	$P(X > 21) = P(Z > 3) = 0.0013$
       
       %e
       \item 
	$P(|X - 6| < 5) = P(-1 < Z < 1) = 0.6826$
       
       %f
       \item 
	$P(|X - 6| < 10) = P(-2 < Z < 2) = 0.9544$
       
       %g
       \item 
	$P(|X - 6| < 15) = P(-3 < Z < 3) = 0.9974$
       
       %h
       \item 
	$P(|X - 6| < 12.41) = P(-2.482 < Z < 2.482) = \Phi(2.482) - \Phi(-2.482) 
	    = 0.9934 - 0.0066 = 0.9868$
      \end{enumerate}
     
     %6
     \item
      $M(t) = exp(166t + 200t^2)$ \\
      $X \sim N(166, 400)$
      \begin{enumerate}
       %a
       \item
	$\mu = 166$
       
       %b
       \item
	$\sigma^2 = 400$
       
       %c
       \item
	$P(170 < X < 200) = P(0.2 < Z < 1.7) = \Phi(1.7) - \Phi(0.2) 
	    = 0.9554 - 0.5793 = 0.3761$
       
       %d
       \item
	$P(148 \le X \le 172) = P(-0.9 \le Z \le 0.3) = \Phi(0.3) - \Phi(-0.9) 
	    = 0.6179 - 0.1841 = 0.4338$
      \end{enumerate}
     
     %7
     \item
      $X \sim N(650, 625)$
      \begin{enumerate}
       %a
       \item 
	$P(600 \le X < 660) = P(-2 \le Z < 0.4) = \Phi(0.4) - \Phi(-2) = 0.6554 - 0.0228 = 0.6326$
       
       %b
       \item
	$P(|X - 650| \le c) = 0.9544$ \\
	$c = 50$
      \end{enumerate}
     
     %8
     \item
      Since $X \sim N(\mu, \sigma^2)$, then the PDF of $X$ is 
	$$f(x) = \dfrac{1}{\sigma \sqrt{2\pi}} exp\left[-\dfrac{(x-\mu)^2}{2\sigma^2}\right]$$
	
      In order to find the points of inflection, we have to find the values of $x$ where the second 
      derivative of the PDF is equal to 0. We start by first computing the first derivative of 
      the PDF:
	\begin{align*}
	  \dfrac{d}{dx}\left(\dfrac{1}{\sigma \sqrt{2\pi}} exp\left[-\dfrac{(x-\mu)^2}{2\sigma^2}\right]\right) 
	    &=\dfrac{1}{\sigma\sqrt{2\pi}} \dfrac{d}{dx}\left(exp\left[-\dfrac{(x-\mu)^2}{2\sigma^2}\right]\right) \\
	  &=\dfrac{1}{\sigma\sqrt{2\pi}} exp\left[-\dfrac{(x-\mu)^2}{2\sigma^2}\right] 
		    \dfrac{d}{dx}\left(-\dfrac{(x-\mu)^2}{2\sigma^2}\right)\\
	  &=\dfrac{1}{\sigma\sqrt{2\pi}} exp\left[-\dfrac{(x-\mu)^2}{2\sigma^2}\right] 
		    \left(-\dfrac{x-\mu}{\sigma^2}\right)\\
	  &=-\dfrac{x-\mu}{\sigma^3\sqrt{2\pi}} exp\left[-\dfrac{(x-\mu)^2}{2\sigma^2}\right] \\
	\end{align*}
	
      We then use that result to calculate the second derivative of the PDF:
	\begin{align*}
	 &\dfrac{d^2}{dx^2}\left(\dfrac{1}{\sigma \sqrt{2\pi}} exp\left[-\dfrac{(x-\mu)^2}{2\sigma^2}\right]\right) \\
	  &= \dfrac{d}{dx}\left(-\dfrac{x-\mu}{\sigma^3\sqrt{2\pi}} exp\left[-\dfrac{(x-\mu)^2}{2\sigma^2}\right]\right) \\
	  &= -\dfrac{x-\mu}{\sigma^3\sqrt{2\pi}} \dfrac{d}{dx}\left(exp\left[-\dfrac{(x-\mu)^2}{2\sigma^2}\right]\right)
	      + exp\left[-\dfrac{(x-\mu)^2}{2\sigma^2}\right] \dfrac{d}{dx}\left(-\dfrac{x-\mu}{\sigma^3\sqrt{2\pi}}\right) \\
	  &= \dfrac{(x-\mu)^2}{\sigma^5\sqrt{2\pi}} exp\left[-\dfrac{(x-\mu)^2}{2\sigma^2}\right] 
	      - \dfrac{1}{\sigma^3 \sqrt{2\pi}} exp\left[-\dfrac{(x-\mu)^2}{2\sigma^2}\right] \\
	  &= \dfrac{(x-\mu)^2 - \sigma^2}{\sigma^5\sqrt{2\pi}} exp\left[-\dfrac{(x-\mu)^2}{2\sigma^2}\right]
	\end{align*}

      We can see that this expression can only be equal to 0 when the first factor is 0,
      since the $exp[...]$ factor is always positive. Therefore, we set the first factor
      equal to 0 and solve for x:
	\begin{align*}
	 \dfrac{(x-\mu)^2 - \sigma^2}{\sigma^5\sqrt{2\pi}} &\seteq 0 \\
	 (x-\mu)^2 - \sigma^2 &= 0 \\
	 (x - \mu)^2 &= \sigma^2 \\
	 x - \mu &= \pm \sigma \\
	 x &= \mu \pm \sigma
	\end{align*}
     
     %9
     \item
      $W = X^2$
      \begin{enumerate}
       %a
       \item 
	$X \sim N(0,4)$ \\
	Cumulative distribution function of $W$:
	$$G(w) = P(W \le w) = P(X^2 \le w) = P(-\sqrt{w} \le X \le \sqrt{w})$$
	
	Let's integrate the PDF of $X$, a normal random variable with $\mu = 0$ and $\sigma = 2$:
	$$G(w) = \mathlarger{\int}_{-\sqrt{w}}^{\sqrt{w}}
	  {\dfrac{1}{2\sqrt{2\pi}}exp\left(-\dfrac{x^2}{8}\right)dx}$$
	  
	We do the following change of variables: Let $x = \sqrt{y} = y^{1/2}$, so 
	$dx = \dfrac{1}{2}y^{-1/2}dy = \dfrac{1}{2\sqrt{y}}dy$. Therefore, $x^2 = y$ and
	$x = 0 \implies y = 0$ and $x = \sqrt{w} \implies y = w$.
	
	This gives us:
	$$G(w) = 2 \mathlarger{\int}_0^{w}{\dfrac{1}{2\sqrt{2\pi}}
		exp\left(-\dfrac{y}{8}\right)\left(\dfrac{1}{2\sqrt{y}}\right)dy}$$
		
	$$G(w) = \mathlarger{\int}_0^{w}{\dfrac{1}{\sqrt{8}\sqrt{\pi}}y^{\frac{1}{2} - 1}
		exp\left(-\dfrac{y}{8}\right)dy}$$
		
	We take the derivative of $G(w)$ to get the probability density function $g(w)$:
	$$g(w) = \dfrac{1}{\sqrt{8}\sqrt{\pi}}y^{\frac{1}{2} - 1}exp\left(-\dfrac{y}{8}\right)$$
	
	which is the gamma distribution PDF with $\alpha=\dfrac{1}{2}$ and $\theta=8$.
	
       %b
       \item
	$X \sim N(0,\sigma^2)$ \\
	Cumulative distribution function of $W$:
	$$G(w) = P(W \le w) = P(X^2 \le w) = P(-\sqrt{w} \le X \le \sqrt{w})$$
	
	Let's integrate the PDF of $X$, a normal random variable with $\mu = 0$ and variance $\sigma^2$:
	$$G(w) = \mathlarger{\int}_{-\sqrt{w}}^{\sqrt{w}}
	  {\dfrac{1}{\sigma\sqrt{2\pi}}exp\left(-\dfrac{x^2}{2\sigma^2}\right)dx}$$
	  
	We do the following change of variables: Let $x = \sqrt{y} = y^{1/2}$, so
	$dx = \dfrac{1}{2}y^{-1/2}dy = \dfrac{1}{2\sqrt{y}}dy$. Therefore, $x^2 = y$ and
	$x = 0 \implies y = 0$ and $x = \sqrt{w} \implies y = w$.
	
	This gives us:
	$$G(w) = 2 \mathlarger{\int}_0^{w}{\dfrac{1}{\sigma\sqrt{2\pi}}
	      exp\left(-\dfrac{y}{2\sigma^2}\right)\left(\dfrac{1}{2\sqrt{y}}\right)dy}$$
	      
	$$G(w) = \mathlarger{\int}_0^{w}{\dfrac{1}{\sqrt{2\sigma^2}\sqrt{\pi}}y^{\frac{1}{2}-1}
	      exp\left(-\dfrac{y}{2\sigma^2}\right)dy}$$
	      
	We take the derivative of $G(w)$ to get the probability density function $g(w)$:
	$$g(w) = \dfrac{1}{\sqrt{2\sigma^2}\sqrt{\pi}}y^{\frac{1}{2} - 1}exp\left(-\dfrac{y}{2\sigma^2}\right)$$
	
	which is the gamma distribution PDF with $\alpha=\dfrac{1}{2}$ and $\theta = 2\sigma^2$.
      \end{enumerate}
     \addtocounter{enumi}{1}
     
     %11
     \item
      $X \sim N(21.37, 0.16)$
      \begin{enumerate}
       %a
       \item 
	$P(X > 22.07) = P(Z > 1.75) = 0.0401$
       
       %b
       \item
	First we compute the probability of a mint weighing less than 20.857 grams, 
	  $$P(X < 20.857) = P(Z < -1.2825) = 0.1003$$
	  
	The distribution of $Y$ is binomial with $n=15$ and $p=0.1003$. We wish to find $P(Y \le 2)$:
	  \begin{align*}
	   P(Y \le 2) &= P(Y = 0) + P(Y = 1) + P(Y = 2) \\
	    &= (1 - 0.1003)^{15} + 15(0.1003)(1 - 0.1003)^14 + 105(0.1003^2)(1 - 0.1003)^13 \\
	    &= 0.8148
	  \end{align*}
      \end{enumerate}
     \addtocounter{enumi}{3}
     
     %15
     \item
      \begin{enumerate}
       %a
       \item 
	If $X \sim N(12.1, \sigma^2)$, then $Z = \frac{X - 12.1}{\sigma}$. Therefore, we can rewrite
	the problem so that we're looking for $c$ such that $P(Z < c) = 0.01$. From Table Va, we know
	that this is the value $-z_{0.01} = -2.326$. Now we can solve for $\sigma$:
	  \begin{align*}
	    -2.326 &= \dfrac{12 - 12.1}{\sigma} \\
	    \sigma &= \dfrac{12 - 12.1}{-2.326} \\
	      &= 0.04299
	  \end{align*}
       
       %b
       \item
	\begin{align*}
	  -2.326 &= \dfrac{12 - \mu}{0.05} \\
	  12 - \mu &= -0.1163 \\
	  \mu &= 12.1163
	\end{align*}

       
      \end{enumerate}
    \end{enumerate}
\end{document}