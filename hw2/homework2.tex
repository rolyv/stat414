\documentclass{article}
    
    
    \usepackage{graphicx} % Used to insert images
    \usepackage{adjustbox} % Used to constrain images to a maximum size 
    \usepackage{color} % Allow colors to be defined
    \usepackage{enumerate} % Needed for markdown enumerations to work
    \usepackage{geometry} % Used to adjust the document margins
    \usepackage{amsmath} % Equations
    \usepackage{amssymb} % Equations
    \usepackage{eurosym} % defines \euro
    \usepackage[mathletters]{ucs} % Extended unicode (utf-8) support
    \usepackage[utf8x]{inputenc} % Allow utf-8 characters in the tex document
    \usepackage{fancyvrb} % verbatim replacement that allows latex
    \usepackage{grffile} % extends the file name processing of package graphics 
                         % to support a larger range 
    % The hyperref package gives us a pdf with properly built
    % internal navigation ('pdf bookmarks' for the table of contents,
    % internal cross-reference links, web links for URLs, etc.)
    \usepackage{hyperref}
    \usepackage{longtable} % longtable support required by pandoc >1.10
    \usepackage{booktabs}  % table support for pandoc > 1.12.2
    \usepackage{indentfirst}
    \usepackage{floatrow}
    
    \newcommand\perm[2]{{}_{#1}P_{#2}}%
    
    
    \title{Homework 2}
    \author{Roly Vicar\'ia \\ STAT414 Spring 2016}
    
\begin{document}
    
    \maketitle
    
    Section 1.2
    \begin{enumerate}
     %1
     \item $8\cdot8\cdot8\cdot8 = 8^4 = 4096$
     \addtocounter{enumi}{1}
     
     %3
     \item 
      \begin{enumerate}
       \item $26\cdot26\cdot10\cdot10\cdot10\cdot10 = 6760000$
       \item $26\cdot26\cdot26\cdot10\cdot10\cdot10 = 17576000$
      \end{enumerate}

      %4
      \item 
	\begin{enumerate}
	 \item $\dbinom{4}{1}\cdot\dbinom{6}{3} = \dfrac{4!}{1!3!}\cdot\dfrac{6!}{3!3!} = 80$
	 \item 
	  \begin{align*}
	    &\dbinom{4}{1}\cdot\left[\dbinom{6}{0}+\dbinom{6}{1}+\dbinom{6}{2}+\dbinom{6}{3}+\dbinom{6}{4}+\dbinom{6}{5}+\dbinom{6}{6}\right] \\
	    &= 4 \cdot (1 + 6 + 15 + 20 + 15 + 6 + 1) \\
	    &= 4 \cdot (54) \\
	    &= 216
	  \end{align*}
	 \item $\dbinom{4-1+3}{3} = \dfrac{(4-1+3)!}{3!(4-1)!} = 20$
	\end{enumerate}

      %5
      \item
	\begin{enumerate}
	 \item $4! = 4\cdot3\cdot2\cdot1 = 24$
	 \item $4\cdot4\cdot4\cdot4 = 4^4 = 256$
	\end{enumerate}
      
      %6
      \item 
	A player can win in 3, 4 or 5 sets, therefore:
	  \begin{align*}
	    &\dbinom{2}{1} \cdot \left[ \dbinom{3}{3} + \dbinom{4}{3} + \dbinom{5}{3} \right] \\
	    &= 2 \cdot (1 + 4 + 10) \\
	    &= 30
	  \end{align*}
	  
      %7
      \item 
	$N(S) = 10^4 = 10000$ 	
	\begin{enumerate}
	 \item 
	    $N(A) = 4! = 24$ \\
	    $P(A) = \dfrac{24}{10000} = 0.0024$
	 \item	    
	    $N(A) = \dfrac{4!}{1!1!2!} = \dfrac{24}{2} = 12$ \\
	    $P(A) = \dfrac{12}{10000} = 0.0012$
	 \item
	    $N(A) = \dfrac{4!}{2!2!} = \dfrac{24}{4} = 6$ \\
	    $P(A) = \dfrac{6}{10000} = 0.0006$
	 \item
	    $N(A) = \dfrac{4!}{1!3!} = \dfrac{24}{6} = 4$ \\
	    $P(A) = \dfrac{4}{10000} = 0.0004$
	\end{enumerate}

      %8
      \item 
	$$ 3\cdot3\cdot\sum_{i=0}^{12}{\dbinom{12}{i}} = 9 \cdot 4096 = 36864 $$
	
      %9
      \item
	\begin{enumerate}
	 \item $\dbinom{2}{1} \cdot \dbinom{4}{4} = 2$
	 \item 3 wins out of first 4 games: $\dbinom{2}{1} \cdot \dbinom{4}{3} = 8$
	 \item 3 wins out of first 5 games: $\dbinom{2}{1} \cdot \dbinom{5}{3} = 20$
	 \item 3 wins out of first 6 games: $\dbinom{2}{1} \cdot \dbinom{6}{3} = 40$
	\end{enumerate}

      %10
      \item
	Starting from the sum and working backwards:
	\begin{align*}
	 \dbinom{n-1}{r} + \dbinom{n-1}{r-1} = \dfrac{(n-1)!}{r!(n-1-r)!} + \dfrac{(n-1)!}{(r-1)!(n-1-r+1)!}
	\end{align*}
	
	In order to add up the terms, we first multiply the first term by $\dfrac{n-r}{n-r}$ and the second
	term by $\dfrac{r}{r}$:	
	\begin{align*}
	 &\dfrac{(n-r)(n-1)!}{r!(n-r)(n-r-1)!} + \dfrac{r(n-1)!}{r(r-1)!(n-r)!} \\
	 &= \dfrac{(n-r)(n-1)! + r(n-1)!}{r!(n-r)!} \\
	 &= \dfrac{(n-1)!(n-r+r)}{r!(n-r)!} = \dfrac{(n-1)!\cdot n}{r!(n-r)!} \\
	 &= \dfrac{n!}{r!(n-r)!} \\
	 &= \dbinom{n}{r}
	\end{align*}
      
      %11
      \item 
	\begin{enumerate}
	 \item $9! = 362880$
	 \item $\dfrac{9!}{3!6!} = 84$
	 \item $2^9 = 512$
	\end{enumerate}

      %12
      \item 
	We can show the first part by restarting 0 as $(1-1)^n$ and applying binomial expansion
	(equation 1.2-1):
	
	\begin{align*}
	 (1 - 1)^n = \sum_{r=0}^n{\dbinom{n}{r}(-1)^r(1)^{n-r}} = \sum_{r=0}^n{(-1)^r\dbinom{n}{r}}
	\end{align*}
	
	We can show the second part by induction and using Pascal's equation
	$$ \sum_{r=0}^n{\dbinom{n}{r}} = 2^n $$
	
	We start with the base case $n = 0$:
	$$ \sum_{r = 0}^n{\dbinom{n}{r}} = \dbinom{0}{0} = 1 = 2^0 $$
	
	Assuming the statement is true for $n-1$, then for all $n > 0$ and $r \in {0,\dots,n}$:
	  \begin{align*}
	   \sum_{r}{\dbinom{n}{r}} &= \sum_{r}{\left[\dbinom{n-1}{r-1} + \dbinom{n-1}{r}\right]} \\
	    &= \sum_{r}{\dbinom{n-1}{r-1}} + \sum_{r}{\dbinom{n-1}{r}} \\
	    &= 2^{n-1} + 2^{n-1} \\
	    &= 2^n
	  \end{align*}
	
      %13
      \item 
	$N(S) = \dbinom{52}{13} = 635013559600$
	\begin{enumerate}
	 \item 
	    $N(A) = \dbinom{13}{5}\dbinom{13}{4}\dbinom{13}{3}\dbinom{13}{1} = 3421322190$ \\
	    $P(A) = \dfrac{3421322190}{635013559600} = 0.00539$
	 \item
	    $N(A) = \dbinom{13}{5}\dbinom{13}{4}\dbinom{13}{2}\dbinom{13}{2} = 5598527220$ \\
	    $P(A) = \dfrac{5598527220}{635013559600} = 0.00882$
	 \item
	    $N(A) = \dbinom{13}{5}\dbinom{13}{4}\dbinom{13}{1}\dbinom{13}{3} = 3421322190$ \\
	    $P(A) = \dfrac{3421322190}{635013559600} = 0.00539$
	 \item
	    No, by the results above, there is a higher probability for hands where the other
	    suits are split 2 and 2. 
	\end{enumerate}

      %14
      \item $\dbinom{10+36-1}{36} = \dfrac{45!}{36!9!} = 886163135$
      
      %15
      \item 
	Equation 1.2-2 states that number of distinguishable permutations of $n$ objects when $n_1$ 
	are similar, $n_2$ are similar, ...., $n_s$ are similar (where $n_1+n_2+\cdots+n_s = n$)
	is equal to:
	$$ \dbinom{n}{n_1,n_2,\cdots,n_s} = \dfrac{n!}{n_1!n_2!\cdots n_s!} $$

	We can arrive at this equation by applying the multiplication rule for each of the $n_s$ 
	subsets:
	$$\dbinom{n}{n_1}\cdot\dbinom{n-n_1}{n_2}\cdots\dbinom{n-n_1-\cdots-n_{s-1}}{n_s}$$
	Expanding this out:
	$$\dfrac{n!}{n_1!(n-n_1)!}\cdot\dfrac{(n-n_1)!}{n_2!(n-n_1-n_2)!}\cdots\dfrac{(n-n_1-n_2-\cdots-n_{s-1})!}{n_s!(n-n_1-n_2-\cdots-n_s)!}$$
	
	We can see that the numerator of every factor after the first cancels out with part of the
	denominator of the factor before it. Also, the denominator of the last factor reduces to
	$n_s!(0!) = n_s!$ since by definition $n_1+n_2+\cdots+n_s = n$. 
	
	Therefore, the previous expression reduces to:
	$$\dfrac{n!}{n_1!n_2!\cdots n_s!}$$
	
      %16
      \item
	$N(S) = \dbinom{52}{9} = 3679075400$
	\begin{enumerate}
	 \item
	    $N(A) = \dbinom{19}{3} \cdot \dbinom{33}{6} = 1073233392$ \\
	    $P(A) = \dfrac{1073233392}{3679075400} = 0.2917$
	 \item 
	    $N(A) = \dbinom{19}{3}\cdot\dbinom{10}{2}\cdot\dbinom{7}{1}\cdot\dbinom{5}{1}\cdot\dbinom{6}{2} = 22892625$ \\
	    $P(A) = \dfrac{22892625}{1073233392} = 0.02133$
	\end{enumerate}

      %17
      \item
	$N(S) = \dbinom{52}{5} = 2598960$
	\begin{enumerate}
	 \item 
	    $N(A) = \dbinom{13}{1}\dbinom{12}{1}\dbinom{4}{1} = 624$ \\
	    $P(A) = \dfrac{624}{2598960} = 0.00024$
	 \item 
	    $N(A) = \dbinom{13}{1}\dbinom{4}{3}\dbinom{12}{1}\dbinom{4}{2} = 3744$ \\
	    $P(A) = \dfrac{3744}{2598960} = 0.00144$
	 \item
	    $N(A) = \dbinom{13}{1}\dbinom{4}{3}\dbinom{12}{2}\dbinom{4}{1}^2 = 54912$ \\
	    $P(A) = \dfrac{54912}{2598960} = 0.02113$
	 \item
	    $N(A) = \dbinom{13}{2}\dbinom{4}{2}^2\dbinom{11}{1}\dbinom{4}{1} = 123552$ \\
	    $P(A) = \dfrac{123552}{2598960} = 0.04754$
	 \item
	    $N(A) = \dbinom{13}{1}\dbinom{4}{2}\dbinom{12}{3}\dbinom{4}{1}^3 = 1098240$ \\
	    $P(A) = \dfrac{1098240}{2598960} = 0.4226$
	\end{enumerate}
      \end{enumerate}
      
      
      Section 1.3
      \begin{enumerate}
        %1
	\item 
	  \begin{enumerate}
	    \item $P(B_1) = \dfrac{5000}{1000000} = 0.005$
	    \item $P(A_1) = \dfrac{78515}{1000000} = 0.078515$
	    \item $P(A_1 | B_2) = \dfrac{73630}{995000} = 0.074$
	    \item $P(B_1 | A_1) = \dfrac{4885}{78515} = 0.0622$
	    \item 
	      Part (c) says that there is a 7.4\% probability of a positive test result given
	      that the person is not carrying the AIDS virus. Part (d) says there is a 6.22% 
	      probability of a carrying the AIDS virus given that the test result was positive. 
	  \end{enumerate}
	  \addtocounter{enumi}{1}
  
	%3
	\item
	  \begin{enumerate}
	    \item $P(A_1 \cap B_1) = \dfrac{5}{35} = 0.1429$
	    \item $P(A_1 \cup B_1) = \dfrac{12 + 19 - 5}{35} = \dfrac{26/35} = 0.7429$
	    \item $P(A_1 | B_1) = \dfrac{5}{19} = 0.2632$
	    \item $P(B_2 | A_2) = \dfrac{9}{23} = 0.3913$
	    \item
	      The event of a student being right-eye-dominant is $A_2$. So if all we can observe
	      is which thumb is on top, left ($B_1$) or right ($B_2$), we choose the one which
	      gives us the highest probability of $A_2$.
	      $P(A_2 | B_1) = \dfrac{14}{19} = 0.7368$ \\
	      $P(A_2 | B_2) = \dfrac{9}{16} = 0.5625$ \\
	      
	      Based on the conditional probabilities, we have a better chance of selecting a 
	      right-eye-dominant student given that their left thumb is on top. 
	  \end{enumerate}

	%4
	\item
	  \begin{enumerate}
	    \item $P(H_1) \cdot P(H_2 | H_1) = \dfrac{13}{52} \cdot \dfrac{12}{51} = 0.0588$
	    \item $P(H_1) \cdot P(C_2 | H_1) = \dfrac{13}{52} \cdot \dfrac{13}{51} = 0.0637$
	    \item We denote drawing a heart first as $H_1$, drawing an ace first 
	      \begin{align*}
		P(H_1 \cap A_2) &= P(H_1 \cap A_1 \cap A_2) + P(H_1 \cap A_1' \cap A_2) \\
		&= P(H_1)\cdot P(A_1|H_1)\cdot P(A_2|H_1 \cap A_1) 
		  + P(H_1)\cdot P(A_1'|H_1)\cdot P(A_2|H_1 \cap A_1') \\
		&= \dfrac{13}{52}\cdot\dfrac{1}{13}\cdot\dfrac{3}{51} 
		+ \dfrac{13}{52}\cdot\dfrac{12}{13}\cdot\dfrac{4}{51} \\
		&= 0.01923
	      \end{align*}
	  \end{enumerate}
	  \addtocounter{enumi}{1}
	  
	%6
	\item 
	  Let H = event that at least one parent had some heart disease \\
	  Let D = event that randomly selected man died of some heart disease \\
	  
	  We are looking for $P(D|H') = \dfrac{P(D \cap H')}{P(H')} 
	  = \dfrac{\dfrac{110}{982}}{\dfrac{648}{982}} = \dfrac{110}{648} = 0.1698$
	  \addtocounter{enumi}{4}
	  
	%11
	\item 
	  \begin{enumerate}
	   \item $365^r$
	   \item $\perm{365}{r} = \dfrac{365!}{(365-r)!}$
	   \item It would be 1 minus the probability of everyone having different birthdays which
	      is part (b) divided by part (a): $1 - \dfrac{\perm{365}{r}}{365^r}$
	   \item 
	      $1 - \dfrac{\perm{365}{r}}{365^r} = .5 \rightarrow r \approx 23$ \\
	      
	      Yes, this number is suprisingly small. In a small room there is already a 50\%
	      chance of two people with the same birthday.
	  \end{enumerate}
	  
	%12
	\item
	  \begin{enumerate}
	    \item 
	      If you draw first: \\
		$P(B_1) = \dfrac{1}{18} = 0.0556$ 
		
	      If you draw fifth: \\
	      \begin{align*}
		&P(R_1 \cap R_2 \cap R_3 \cap R_4 \cap B_5) \\
		&= P(R_1)\cdot P(R_2|R_1)\cdot P(R_3|R_1 \cap R_2)
		  P(R_4|R_1 \cap R_2 \cap R_3)\cdot P(B_5|R_1 \cap R_2 \cap R_3 \cap R_4) \\
		&= \dfrac{17}{18}\cdot\dfrac{16}{17}\cdot\dfrac{15}{16}\cdot\dfrac{14}{15}\cdot\dfrac{1}{14} \\
		&= 0.0556
	      \end{align*}
	      
	      If you draw last: \\
	      \begin{align*}
	       &P(R_1 \cap R_2 \cap \cdots \cap R_{17} \cap B_{18}) \\
	       &= P(R_1)P(R_2|R_1)\cdots P(R_{17}|R_1 \cap R_2 \cap \cdots \cap R_{16}) 
		P(B_{18}|R_1 \cap R_2 \cap \cdots \cap R_{17}) \\
	       &= \dfrac{17}{18} \cdot \dfrac{16}{17} \cdots \dfrac{1}{2} \cdot \dfrac{1}{1} \\
	       &= 0.0556
	      \end{align*}

	      You have the same chance of drawing the blue pebble regardless which position we choose.
	    \item 
	      If you draw first: \\
		$P(B_1) = \dfrac{2}{18} = 0.1111$ 
		
	      If you draw fifth: \\
	      \begin{align*}
		&P(R_1 \cap R_2 \cap R_3 \cap R_4 \cap B_5) \\
		  &+ P(B_1 \cap R_2 \cap R_3 \cap R_4 \cap B_5) \\
		  &+ P(R_1 \cap B_2 \cap R_3 \cap R_4 \cap B_5) \\
		  &+ P(R_1 \cap R_2 \cap B_3 \cap R_4 \cap B_5) \\
		  &+ P(R_1 \cap R_2 \cap R_3 \cap B_4 \cap B_5) \\
		&= \dfrac{16}{18}\cdot\dfrac{15}{17}\cdot\dfrac{14}{16}\cdot\dfrac{13}{15}\cdot\dfrac{2}{14} \\
		  &+ 4 \cdot 
		    \left(\dfrac{16}{18}\cdot\dfrac{15}{17}\cdot\dfrac{14}{16}\cdot\dfrac{2}{15}\cdot\dfrac{1}{14}\right) \\
		&= 0.1765
	      \end{align*}
	      
	      If you draw last, the probability reduces to the probability of drawing 1 blue pebble
	      out of the 18 pebbles:$\dfrac{1}{18} = 0.0556$. 
	      
	      Therefore, in the case of 2 blue pebbles, I choose to draw 5th. 
	  \end{enumerate}
	  \addtocounter{enumi}{1}
	  
	%14
	\item 
	  \begin{enumerate}
	   \item $P(A_1) = \dfrac{7+11+12}{100} = 0.3$
	   \item $P(A_3 \cap B_2) = \dfrac{9}{100} = 0.09$
	   \item $P(A_2 \cup B_3) = \dfrac{11+21+9+12+7}{100} = 0.6$
	   \item $P(A_1 | B_2) = \dfrac{11}{11+21+9} = 0.2683$
	   \item $P(B_1 | A_3) = \dfrac{13}{13+9+7} = 0.4483$
	  \end{enumerate}
	  \addtocounter{enumi}{1}
	  
	%16
	\item 
	  We define:\\
	    RA as drawing a red chip from bowl A \\
	    RB as drawing a red chip from bowl B \\
	    WA as drawing a white chip from bowl A 
	    
	  \begin{align*}
	    P(RB) &= P(RA)\cdot P(RB|RA) + P(WA)\cdot P(RB|WA) \\
		  &= \dfrac{3}{5}\cdot\dfrac{5}{8} + \dfrac{2}{5}\cdot\dfrac{4}{8} \\
		  &= 0.575
	  \end{align*}

      \end{enumerate}
 
\end{document}
